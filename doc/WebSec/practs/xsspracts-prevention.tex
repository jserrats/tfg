\begin{Exercise}[label={websec-xss-types}]
	Modify the source code of all three of the previous exercises so they are no longer vulnerable to a basic attack.
\end{Exercise}
\begin{Answer}[ref={websec-xss-types}]
	\subsection{Stored XSS}
	Due to the fact that the application does not have any kind of XSS protection, the injection is as easy as typing <script>alert</script> in any of the form fields. The next user that loads the comments will be affected by this XSS.
	\subsection{Reflected XSS}
	As we can see, the code parses the GET request parameters directly into the application. To trigger an alert() we only would have to send the victim an URL like: http://localhost:8000/search?searchquery=<script>alert()</script>
	\subsection{DOM based XSS}
	AS before, the answer is to write <script>alert()</script> into the form.
\end{Answer}