
\begin{Exercise}[label={websec-xss-types}]
	\subsubsection{Stored XSS}
	In this exercise we will practice finding vulnerabilities in a very basic web application.
	
	Run the following node code in your machine. Open a browser and go to http://localhost:8000. While looking at the source code and developer options of the browser, find a way to execute a alert() function. The alert() function makes the browser send a popup. For this exercise we will consider that if we can run an alert() the application is compromised.
	
	\begin{lstlisting}[style=JavaScript]
	var express = require('express');
	var bodyParser = require('body-parser');
	var app = express();

	app.use(bodyParser.urlencoded({extended: true})); // for parsing application/x-www-form-urlencoded
	
	db = [];
	
	app.get('/', function (req, res) {
	list = "<ul>";
	for (var i = 0; i < db.length; i++) {
	list += "<li>" + db[i].user + " said: " + db[i].comment + "</li>"
	}
	list += "</ul>";
	
	res.send('<html>' +
	'<head>' +
	'<title>The cat forum</title>' +
	'</head>' +
	'<body>' +
	"<h1>The Cat Forum. A place to talk about cats. </h1>" +
	list +
	"<form action=\"/comment\" method=\"POST\"><div><label>User</label><input type=\"text\" name=\"user\"/></div><div><label>Comment</label><input type=\"text\" name=\"comment\"/></div><div><input type=\"submit\" value=\"Post your comment\" size=\"100\"/></div></form>" +
	'</body>' +
	'</html>')
	});
	
	app.post('/comment', function (req, res) {
	db.push({comment: req.body.comment, user: req.body.user});
	res.redirect("/");
	});
	
	app.listen(8000, () => console.log('Listening on 8000'));
	\end{lstlisting}
	
	\subsubsection{Reflected XSS}
		Do the same with the following application.
	\begin{lstlisting}[style=JavaScript]
	var express = require('express');
	var app = express();
	
	db = {
	"apples": 2,
	"watermelons": 5,
	"pineapples": 3
	};
	
	app.get('/', function (req, res) {
	res.send("<html><head>" +
	"<title>Fruit stock tracker</title></head>" +
	"<body><h1>Write the name of the fruit to know how many we have</h1>" +
	"<div><form action=\"/search\" method=\"GET\">" +
	"<input type=\"text\" name=\"searchquery\"/>" +
	"<input type=\"submit\" value=\"Submit\"/>" +
	"</form></div>" +
	"</body></html>")
	});
	
	
	app.get('/search', function (req, res) {
	var query = req.query.searchquery;
	var result = db[query];
	res.send('<html>' +
	'<head>' +
	'<title>Fruit stock tracker</title>' +
	'</head>' +
	'<body>' +
	'We have ' + result + ' ' + query +
	'</body>' +
	'</html>')
	});
	app.listen(8000, () => console.log('Listening on 8000'));
	\end{lstlisting}
	

	\subsubsection{DOM based XSS}
		Do the same with this code. Remember to put the html in a file called 'index.html' in the same folder. This exercise is very similar to the reflected xss, but the important thing to understand is the difference between them.
		\begin{lstlisting}[style=JavaScript]
		const express = require('express');
		var path = require('path');
		const app = express();
		
		app.get('/', (req, res) => res.sendFile(path.join(__dirname + '/index.html')));
		
		app.listen(8000, () => console.log('Example app listening on port 8000!'));
		\end{lstlisting}
		\begin{lstlisting}[style=JavaScript]
			<!DOCTYPE html>
			<html>
			<body>
			
			<form id="myForm">
			What is your name?: <input type="text"><br>
			</form>
			<button onclick="getName()">Done</button>
			<p id="output"></p>
			
			<script>
			function getName(){
			var x = document.getElementById("myForm").elements[0].value;
			document.getElementById("output").innerHTML = "Hello, " + x;
			}
			</script>
			
			</body>
			</html>
		\end{lstlisting}
\end{Exercise}
\begin{Answer}[ref={websec-xss-types}]
	\subsubsection{Stored XSS}
	Due to the fact that the application does not have any kind of XSS protection, the injection is as easy as typing <script>alert()</script> in any of the form fields. The next user that loads the comments will be affected by this XSS.
	\subsubsection{Reflected XSS}
	As we can see, the code parses the GET request parameters directly into the application. To trigger an alert() we only would have to send the victim an URL like: http://localhost:8000/search?searchquery=<script>alert()</script>
	\subsubsection{DOM based XSS}
	AS before, the answer is to write <script>alert()</script> into the form.
\end{Answer}