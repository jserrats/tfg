%Author:jaume,jose
%Last:03/2018
%Version:1.0
%Comment: basic RSA with JS

\begin{Exercise}[label={basic-js-crypto-rsa}]
In this exercise you will practice with asymmetric encryption.
First, generate a key pair using openssl:

\begin{lstlisting}[style=terms]
$ openssl genrsa -out key.priv 2048 && openssl rsa -in key.priv -pubout -out key.pub
\end{lstlisting}

When finished, we will have two files, 
one with the public key and the other one with the private.

\begin{enumerate}[1.]
	
\item Write a simple script that loads the keys from the files 
and encrypts and decrypts a string using RSA using 
the library \textit{node-rsa.}
Use the following snippets as a guide:
\begin{js}
file_content = fs.readFileSync('file.txt');
private_key = new NodeRSA(" PEM formatted key ");
encrypted_data = public_key.encrypt(data, 'base64');
private_key.decrypt(data, 'utf8');
\end{js}
\item Now use the signing capability of RSA. Use the sign and verify methods:
\begin{js}
key.sign(buffer, [encoding], [source_encoding]);
key.verify(buffer, signature, [source_encoding], [signature_encoding])
\end{js}
	
\end{enumerate}
\end{Exercise}

\begin{Answer}[ref={basic-js-crypto-rsa}]
\begin{enumerate}[1.]
\item The code is the following: 

\begin{js}
// openssl genrsa -out certs/server/my-server.key.pem 2048
// openssl rsa -in certs/source/my-source.key.pem -pubout -out certs/destination/my-source.pub
var fs = require('fs');
const NodeRSA = require('node-rsa');

private_key = new NodeRSA(fs.readFileSync('key.priv'));
public_key = new NodeRSA(fs.readFileSync('key.pub'));


data = "testaedaedaedae";

console.log(private_key.decrypt(public_key.encrypt(data, 'base64'), 'utf8'));

signed = private_key.sign(data, 'base64');

console.log(public_key.verify(data, signed, 'utf8', 'base64'));
\end{js}
\end{enumerate}
\end{Answer}
