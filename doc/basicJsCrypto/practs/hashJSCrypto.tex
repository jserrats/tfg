%Author:jaume,jose
%Last:03/2018
%Version:1.0
%Comment: basic SHA256 with JS

\begin{Exercise}[label={basic-js-crypto-sha256}]
In this exercise you will practice with cryptographic hashes.

\begin{enumerate}[1.]
\item Write a script that prints the hash of a string using SHA265. 
Use the \textit{crypto} module of the Node standard library.
\item Modify the script so that it also prints 
the hash of a slightly different string. 
\item Compare the two results. 
Do you think it's easy to find a collision? 
How many possible hash digests are there in SHA256?
	
\end{enumerate}
\end{Exercise}

\begin{Answer}[ref={basic-js-crypto-sha256}]
\begin{enumerate}[1.]
\item The code is the following: 

\begin{lstlisting}[style=JavaScript]
// import the nodejs standard library "crypto"
const crypto = require('crypto');


function print_hash(message) {
    console.log("\nMessage: " + message +
        "\nHASH: " + 
        crypto.createHash('sha256').update(message).digest('hex'))
}

print_hash("Bob owes me 100$");
print_hash("Bob owes me 900$");

console.log("\nThere are " + Math.pow(2, 256) 
+ " possible sha256 outputs. Collisions are possible but very unlikely");
\end{lstlisting}
\end{enumerate}
\end{Answer}


%Author:jaume,jose
%Last:03/2018
%Version:1.0
%Comment: basic bcrypt with JS

\begin{Exercise}[label={basic-js-crypto-bcrypt}]
In this exercise you will practice with the bcrypt algorithm for hashing passwords.

\begin{enumerate}[1.]
\item Write a script that asks the user for a password, stores it in memory as a bcrypt, and then asks again for the password. The script has to compare both bcrypts and login the user. Use the \textit{bcrypt} library for Node.

\begin{lstlisting}[style=JavaScript]
bcrypt.hash(myPlaintextPassword, saltRounds, function(err, hash) {
// Store hash in your password DB.
});

bcrypt.compare(myPlaintextPassword, hash, function(err, res) {
// res == true
});

bcrypt.compare(someOtherPlaintextPassword, hash, function(err, res) {
// res == false
});
\end{lstlisting}

Use this snippet for making easy password prompts:
\begin{lstlisting}[style=JavaScript]
function ask_something(to_ask) {
	return new Promise((resolve) => {
		rl.question(to_ask, (answer) => {
			resolve(answer);
		});
	});
}

async function main() {
	const password = await ask_something('Write a password to be saved: ');
}
\end{lstlisting}
\end{enumerate}
\end{Exercise}

\begin{Answer}[ref={basic-js-crypto-bcrypt}]
\begin{enumerate}[1.]
\item The code is the following: 

\begin{lstlisting}[style=JavaScript]
const bcrypt = require('bcrypt');
const readline = require('readline')
const rl = readline.createInterface(process.stdin, process.stdout);

stored_hash = '';

function ask_something(to_ask) {
    return new Promise((resolve) => {
        rl.question(to_ask, (answer) => {
            resolve(answer);
        });
    });
}

async function main() {

    const password = await ask_something('Write a password to be saved: ');

    bcrypt.hash(password, 10, function (err, hash) {
        stored_hash = hash;
    });

    const new_password = await ask_something('Now write the password to login: ');

    console.log("The bcrypt stored is: " + stored_hash);

    bcrypt.compare(new_password, stored_hash, function (err, res) {

        if (res) {
            console.log("The passwords are the same, log in!")
        } else {
            console.log("The passwords are diferent")
        }
    });
    rl.close();
    process.stdin.destroy();
}

main();
\end{lstlisting}
\end{enumerate}
\end{Answer}
