%Author:jaume,jose
%Last:03/2018
%Version:1.0
%Comment: basic AES with JS

\begin{Exercise}[label={basic-js-crypto-aes}]
In this exercise you will practice with symmetric encryption.

\begin{enumerate}[1.]
\item Write a simple script that encrypts and decrypts a string using ECB.
In your script, prints to the screen the whole process. 
The password and the string to be encrypted can be hard coded.
\end{enumerate}

\end{Exercise}

\begin{Answer}[ref={basic-js-crypto-aes}]
\begin{enumerate}[1.]
\item The code is the following: 

\begin{lstlisting}[style=JavaScript]
var crypto = require('crypto');
password = 'asdfe';

//https://en.wikipedia.org/wiki/Block_cipher_mode_of_operation

function encrypt(text) {
    var cipher = crypto.createCipher('aes-256-ecb', password);
    var crypted = cipher.update(text, 'utf8', 'hex');
    crypted += cipher.final('hex');

    return crypted;
}

function decrypt(text) {
    var decipher = crypto.createDecipher('aes-256-ecb', password);
    var dec = decipher.update(text, 'hex', 'utf8');
    dec += decipher.final('utf8');
    return dec;
}

message = "01234567890123456789012345678901234567890123456";
var hw = encrypt(message);
split = hw.toString().length / 2;
console.log("Message is " + message.length + " bytes long. " +
    "Encrypted is " + hw.toString().length + " chars or " + hw.length /2 + "bytes");
console.log(hw, "\n" + hw.substr(0, split), "\n" + hw.substr(split));
console.log(decrypt(hw));
\end{lstlisting}

\end{enumerate}
\end{Answer}
