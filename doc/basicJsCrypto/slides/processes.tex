\section{Processes}
\subsection{Introduction}
\begin{frame}[fragile]\frametitle{Booting}
\begin{itemize}
\item To boot (start) a Linux system, a sequence is followed in which 
the control first goes to the BIOS, then to a boot loader 
and finally, to a Linux kernel (the system core).
\begin{figure}
\includegraphics [width=0.85\textwidth]{\linuxdir/processes/figures/boot-process}
\end{figure}
\item When kernel starts, it executes \textit{init}, the first process. 
\item The kernel implements a ''scheduler`` to share the CPU.
\item Linux processes have ''kinship`` (parent, child etc.). 
\item The root of the ''tree of processes`` is \textit{init}. 
\end{itemize}
\end{frame}

\begin{frame}[fragile]\frametitle{Listing processes}
\begin{itemize}
\item The command \verb|ps| provides information about the processes running on the system.

\begin{lstlisting}
$ ps
  PID TTY          TIME CMD
21380 pts/3    00:00:00 bash
21426 pts/3    00:00:00 ps
\end{lstlisting}

\item We see that two processes \verb|bash| (shell) and \verb|ps| (command). 
\item The PID is the process identifier.
\end{itemize}
\begin{figure}
\includegraphics [width=0.85\textwidth]{\linuxdir/processes/figures/te-processes}
\end{figure}   
\end{frame}

\begin{frame}[fragile]\frametitle{Command \texttt{ps}}
\begin{itemize}
\item \texttt{\textbf{ps}}  supports many parameters. 
\item Some of them are (type \texttt{man ps}):
  \begin{itemize}
  \scriptsize
  \item \textit{\textbf{-A}} shows all the processes from all the users. \\
  \item \textit{\textbf{-u user}} shows proccesses of a particular user. \\
  \item \textit{\textbf{-f}} shows extended information. \\
  \item \textit{\textbf{-o format}} format may be included in a list separated by commas
  the columns of information you want displayed (use the command \texttt{man}
  for a complete list of possible columns). 
  \item Examples: 
  \end{itemize}

\scriptsize
\begin{lstlisting}
$ ps -Ao  pid,ppid,state,tname,%cpu,%mem,time,cmd
$ ps -u user1 -o pid,ppid,cmd
\end{lstlisting}
\end{itemize}
\end{frame}

\begin{frame}[fragile]\frametitle{Manual}
\begin{itemize}
 \item The \verb|man| command shows the ``manual'' of other commands. 
\begin{lstlisting}
$ man ps
\end{lstlisting}
\item Manual for the command ``\texttt{ps}''. 
\item Use arrow keys or \texttt{AvPag}/\texttt{RePaq} to go up and down. 
\item To search for text \textit{xxx} 
  \begin{itemize}
  \item You can type \texttt{/xxx}. 
  \item To go to the next and previous matches you can press keys
  \texttt{n} and \texttt{p} respectively. 
  \end{itemize}
\item You can use \texttt{q} to exit the manual.
\end{itemize}
\end{frame}

\begin{frame}[fragile]\frametitle{Working with the terminal}
\begin{itemize}
\item \textbf{History:}
  \begin{itemize}
  \item You can also press the up/down arrow to scroll back 
  and forth through your command history.
  \item The history can be seen with the command \texttt{history} 
  and you can retype a command with \texttt{!number}.
  \end{itemize}
\item \textbf{Completition:} 
  \begin{itemize}
  \item When pressing the \texttt{TAB}, bash automatically fills 
  in partially typed commands or parameters.
  \item Example: type h+TAB, h+TAB+TAB, hi+TAB+TAB, etc.
  \end{itemize}
\item \textbf{Copy and Paste:} 
  \begin{enumerate}
  \item Select text and press the mouse's middle button (or scroll wheel) to paste.
  \item The combinations \texttt{CRL+SHIFT+c} and \texttt{CRL+SHIFT+v} also usually work.
  \end{enumerate}
\end{itemize}
\end{frame}

\begin{frame}\frametitle{Other commands related to processes}
\begin{itemize}
\item \texttt{\textbf{pstree}} displays all system processes tree.
\item \texttt{\textbf{top}} returns a list of processes with information 
updated periodically. 
\item \texttt{\textbf{time}} gives us the duration of execution 
of a particular command.
  \begin{itemize}
  \item Real refers to actual elapsed time. 
  \item User and Sys refer to CPU time used only by the process.
  \end{itemize}
\end{itemize}
\end{frame}

\subsection {Scripts}

\begin{frame}[fragile]\frametitle{Script vs. Program}
\begin{columns}
\begin{column}{0.55\textwidth}
\begin{itemize}
\item \textbf{Programs:}
  \begin{itemize}
  \item The source of a program is first compiled, and the result 
  of that compilation is executed.
  \end{itemize}
\end{itemize}
\end{column}
\begin{column}{0.5\textwidth}
\begin{figure}
\includegraphics [width=1\textwidth]{\linuxdir/processes/figures/program-script}
\end{figure}
\end{column}
\end{columns}
\begin{itemize}
\small
\item Examples of languages to build programs: C, C++, etc.
\end{itemize}

\vspace{-0.2cm}
\begin{itemize} 
\item \textbf{Scripts:}
  \begin{itemize}
  \item A script is interpreted. It is written to be understood by an interpreter.
  \item Scripting examples: Bash scripts, Python, PHP, Javascript, etc.
  \item Typically scripts are written for small applications and they are easier to develop.
  \item However, scripts are also usually slower than programs due to the interpretation process.
  \end{itemize}
\end{itemize}
\end{frame}

\begin{frame}[fragile,allowframebreaks]\frametitle{Shell Scripts}
\begin{itemize}
\item A shell script is a text file containing commands and special 
internal shell commands (if,for, while, etc.).
\item The script is interpreted and executed by the shell (bash in most Linux systems). 
\item The simplest example is:
\begin{lstlisting}[style=scriptStyle]
pstree
sleep 2
ps
\end{lstlisting}

\item To run a script you must give it execution permissions: 
\begin{lstlisting}
$ chmod u+x myscript.sh 
\end{lstlisting}
\item To execute it use: 
\begin{lstlisting}
$ ./myscript.sh
\end{lstlisting}
\framebreak
\item Another example script is the typical ``Hello world'': 
\begin{lstlisting}[style=scriptStyle]
#!/bin/bash
# Our second script, Hello world!
echo Hello world
\end{lstlisting}
\item The script begins with ``\#!'' which contains the path to the shell that will execute the script.
\item The lines starting with \# are comments.
\item To write to the terminal we can use the \texttt{echo} command.
\item To read you can use the \texttt{read} command.
\begin{lstlisting}[style=scriptStyle]
#!/bin/bash
# Our third script, using read for fun
echo Please, type a sentence and hit ENTER
read TEXT
echo You typed: $TEXT
\end{lstlisting}
\end{itemize}
\end{frame}

\subsection {Fore/Background}

\begin{frame}[fragile]\frametitle{Foreground and Background}
\begin{itemize}
\item By default, the bash executes commands interactively or \textbf{foreground}.
\item The shell waits until the end of a command before executing another one.
\begin{lstlisting}
$ xeyes
\end{lstlisting}
\item With the ampersand symbol (\&), you can execute commands non-interactively or in \textbf{background}.
\begin{lstlisting}
$ xeyes &
\end{lstlisting}
\item In foreground or background the output goes to the corresponding terminal.
\item You cannot use input from the terminal while in background.
\end{itemize}
\end{frame}

\subsection {Signals}
\begin{frame}[fragile,allowframebreaks]\frametitle{Signals}
\begin{itemize}
\item A signal is a limited form of inter-process communication: signals are INTEGERS.
\item Some signals are destined to the kernel (non-capturable) and others to processes running in user space (capturable).
\begin{figure}
\begin{centering}
\includegraphics [width=0.7\textwidth]{\linuxdir/processes/figures/kill-syscall}
\par\end{centering}
\end{figure}
\item When the signal is for a process it can be understood as an asynchronous notification.
\framebreak
\item When the process receives the signal it interrupts its normal flow of execution and
it executes the corresponding signal handler (function).
\item In Linux, the most widely used signals and their corresponding integers are:
  \begin{itemize}
  \item 9 \verb|SIGKILL|. Non-capturable signal sent to the kernel to end a process immediately.
  \item 20 \verb|SIGSTOP|. Non-capturable signal sent to the kernel to stop a process. This signal can be 
  generated in a terminal for a process in foreground pressing \textbf{\texttt{Control-Z}}.
  \item 18 \verb|SIGCONT|. Non-capturable signal sent to the kernel that resumes a previously stopped process. This 
  signal can be generated typing \texttt{bg} in a terminal.
  \framebreak
  \item 2 \verb|SIGINT|. Capturable signal sent to a process to tell it that it must terminate its execution. 
  It is sent in an interactive terminal for the process in foreground when the user presses \textbf{\texttt{Control-C}}.
  \item 15 \verb|SIGTERM|. Capturable signal sent to a process to ask for termination. It is sent from the GUI and 
  also this is the default signal sent by the \texttt{kill} command.
  \item \verb|USR1|. Capturable signal that can be used for any desired purpose.
  \end{itemize}
\item Syntax of \verb|kill| command: \textbf{\texttt{kill -signal PID}}. 
\begin{lstlisting}
$ kill -9 30497
$ kill -SIGKILL 30497 
\end{lstlisting}
\item As you can observe, you can use both the number and the name of the signal.
\end{itemize}
\end{frame}

% Lanzar un proceso en foreground y usar stop/cont para ponerlo en background.
% Por lo complejo de la operación justificamos el Job Control.

\begin{frame}[fragile,allowframebreaks]\frametitle{Job Control}
\begin{itemize}
\item \textit{Job control} refers to the bash feature of managing processes as jobs.
\item We use \textbf{\texttt{``jobs''}}, \textbf{\texttt{``fg''}},  \textbf{\texttt{``bg''}} 
and the hot keys \textbf{\texttt{Control-z}} and \textbf{\texttt{Control-c}}.
\item \texttt{jobs} displays a list of processes launched from a specific instance of bash. 
\item Each job is assigned an identifier called a JID (Job Identifier).
\item \textbf{Control-z} sends a stop signal (SIGSTOP) to the process that is running on \textit{foreground}. 
\item To resume the process that we just stopped, type the command \texttt{bg}.
\item Typping the JID after the command \texttt{bg} will send the process identified by it to \textit{background}. 
\item The JID can also be used with the command \texttt{kill} using \%.
\item Another very common shortcut is \textbf{Control-c} and it is used to send a
signal to terminate (SIGINT) the process that is running on
\textit{foreground}. 
\item Whenever a new process is run in \textit{background}, the bash provides us
the JID and the PID:
\begin{lstlisting}
$ xeyes &
[1] 25647 
\end{lstlisting}
\item Here, the job has JID=1 and PID=25647.
\end{itemize}
\end{frame}

\begin{frame}[fragile]\frametitle{\texttt{trap}}
\begin{itemize}
\item \texttt{trap} allows capturing and processing signals in scripts.
\item Example, if we use this script:

\begin{lstlisting}[style=scriptStyle]
trap "echo I do not want to finish!!!!" SIGINT
while true
do
sleep 1
done
\end{lstlisting}
\item Try to press \texttt{\textbf{Control-z}}.
\end{itemize}
\end{frame}

\ifEXTRA
\begin{frame}[fragile]\frametitle{\texttt{*nice}}
\begin{itemize}
\item Each Unix process has a priority level ranging from -20 (highest priority) 
to 19 (lowest priority).
\item A low priority means that the process will run more slowly or that the process 
will receive less CPU cycles from the kernel scheduler. 
\item The \verb|top| command can be used to easily change the priority of running processes.
\item Only the superuser ''root`` can assign negative priority values.
\item You can also use \texttt{nice} and \texttt{renice} instead of \texttt{top}. 
Examples.
\begin{lstlisting}
$ nice -n 2 xeyes &
[1] 30497
$ nice -n -2 xeyes
nice: cannot set niceness: Permission denied
$ renice -n 8 30497
30497: old priority 2, new priority 8
\end{lstlisting}
\end{itemize}
\end{frame}
\fi


\subsection{Multiple commands}

\begin{frame}[fragile,allowframebreaks]\frametitle{Running multiple commands}
\begin{itemize}
\item The commands can be run in some different ways. 
\item In general, the command returns 0 if
successfully executed and positive values (usually 1) if an error occurred.
\item To see the exit status type \texttt{echo \$?}. 
\item Try:

\begin{lstlisting}
$ ps -k
$ echo $?
$ ps 
$ echo $?
\end{lstlisting}

\framebreak
\item There are also different ways of executing commands:
  \begin{itemize}
  \item \textbf{\texttt{\$~command~}} the command runs in the foreground. 
  \item \textbf{\texttt{\$~command1~\&~command2~\&~...~commandN~\&}} commands will run in background.
  \item \textbf{\texttt{\$~command1;~command2~... ;~commandN}} sequential execution.
  \item \textbf{\texttt{\$~command1~\&\&~command2~\&\&~...~\&\&~commandN}} \texttt{commandX} is executed
  if the last executed command has exit successfully (return code 0).
  \item \textbf{\texttt{\$~command1~||~command2~||~...~||~commandN}} \texttt{commandX} is executed
  if the last executed command has NOT exit successfully (return code >0).
  \end{itemize}
\end{itemize}
\end{frame}